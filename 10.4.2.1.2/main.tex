\let\negmpace\undefined
\let\negthickspace\undefined
\documentclass[journal]{IEEEtran}
\usepackage[a5paper, margin=10mm, onecolumn]{geometry}
%\usepackage{lmodern} % Ensure lmodern is loaded for pdflatex
\usepackage{tfrupee} % Include tfrupee package
\setlength{\headheight}{1cm} % Set the height of the header box
\setlength{\headsep}{0mm}     % Set the distance between the header box and the top of the text
\usepackage{xparse}
\usepackage{gvv-book}
\usepackage{gvv}
\usepackage{cite}
\usepackage{amsmath,amssymb,amsfonts,amsthm}
\usepackage{algorithmic}
\usepackage{graphicx}
\usepackage{textcomp}
\usepackage{xcolor}
\usepackage{txfonts}
\usepackage{listings}
\usepackage{enumitem}
\usepackage{mathtools}
\usepackage{gensymb}
\usepackage{comment}
\usepackage[breaklinks=true]{hyperref}
\usepackage{tkz-euclide} 
\usepackage{listings}
% \usepackage{gvv}                                        
\def\inputGnumericTable{}                                 
\usepackage[latin1]{inputenc}                                
\usepackage{color}                                            
\usepackage{array}                                            
\usepackage{longtable}                                       
\usepackage{calc}                                             
\usepackage{multirow}                                         
\usepackage{hhline}                                           
\usepackage{ifthen}                                           
\usepackage{lscape}
\renewcommand{\thefigure}{\theenumi}
\renewcommand{\thetable}{\theenumi}
\setlength{\intextsep}{10pt} % Space between text and floats
\numberwithin{equation}{enumi}
\numberwithin{figure}{enumi}
\renewcommand{\thetable}{\theenumi}
\begin{document}
\bibliographystyle{IEEEtran}
\title{Question-10.4.2.1.2}
\author{EE24BTECH11038 - MALAKALA BALA SUBRAHMANYA ARAVIND}
% \maketitle
% \newpage
% \bigskip
{\let\newpage\relax\maketitle}
\textbf{Question}:
Find the roots of the equation $2x^2+x-6=0$\\
\solution \\
We can solve the above equation using fixed point iterations. First we separate $x$, from the above equation and make an update equation of the below sort.
\begin{align}
	x = g\brak{x} \implies x_{n+1} = g\brak{x_n}
\end{align}
Applying the above update equation on our equation, we get
\begin{align}
    x_{n+1}=6-2x_n^2\\
\end{align}
Now we take an initial value $x_0$ and iterate the above update equation. But we realize that the updated values always approach infinity for any initial value. \\
Thus we will alternatively use Newton's Method for solving equations.
\begin{align}
	x_{n+1} = x_n - \frac{f\brak{x_n}}{f^{\prime}\brak{x_n}} 
\end{align}
Where we define $f\brak{x}$ as, 
\begin{align}
	f\brak{x} = 2x^2+x-6 \\
	f^{\prime}\brak{x} = 4x+1
\end{align}
Thus, the new update equation is, 
\begin{align}
	x_{n+1} = x_n - \frac{2x_n^2+x_n-6}{4x_n+1 } 
\end{align}
Taking an initial guess of $x_0=-4$ we can see that $x_n$ converges at the 3rd iteration with x as
\begin{align}
    x=-2
\end{align}
Taking an initial guess of $x_0=1$ we can see that $x_n$ converges at the 3rd iteration with x as
\begin{align}
    x=\frac{3}{2}
\end{align}
\begin{figure}[h!]
	\centering
	\includegraphics[width=\columnwidth]{figs/Figure_1.png}
	\label{stemplot}
\end{figure}
\end{document}
